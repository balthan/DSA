% preface.tex
% A Practical Introduction to Data Structures and Algorithm Analysis
% 3rd Edition: Shared between C++ and Java versions

% This next line gets the preface into the TOC without
% incrementing the chapter numbering.
\addcontentsline{toc}{chapter}{Preface}
\chapter*{Preface}
\label{Preface}

We study data structures so that we can learn to write more efficient
programs.\index{efficiency}
But why must programs be efficient when new computers are faster every
year?
The reason is that our ambitions grow with our capabilities.
Instead of rendering efficiency needs obsolete, the modern revolution
in computing power and storage capability merely raises the efficiency
stakes as we attempt more complex tasks.

The quest for program efficiency need not and should not conflict with
sound design and clear coding.
Creating efficient programs has little to do with ``programming
tricks'' but rather is based on good organization of information and
good algorithms.
A programmer who has not mastered the basic principles of clear design
is not likely to write efficient programs.
Conversely, concerns related to development costs and maintainability
should not be used as an excuse to justify inefficient performance.
Generality in design can and should be achieved without sacrificing
performance, but this can only be done if the designer understands how
to measure performance and does so as an integral part of the design
and implementation process.
Most computer science curricula recognize that good programming
skills begin with a strong emphasis on fundamental software
engineering principles.\index{software engineering}
Then, once a programmer has learned the principles of clear
program design and implementation, the next step is to study
the effects of data organization and algorithms on program efficiency.

\paragraph{Approach:}
This book describes many techniques for representing data.
These techniques are presented within the context of
the following principles:

\begin{enumerate}

\item
Each data structure and each algorithm has costs and
benefits.\index{data structure!costs and benefits}
Practitioners need a thorough understanding of how to assess costs and
benefits to be able to adapt to new design challenges.
This requires an understanding of the principles of
algorithm analysis,\index{algorithm analysis}
and also an appreciation for the significant effects of the physical
medium employed (e.g., data stored on disk versus main memory).

\item
Related to costs and benefits is the notion of
tradeoffs.\index{tradeoff}
For example, it is quite common to reduce time requirements at the
expense of an increase in space requirements, or vice versa.
Programmers face tradeoff issues regularly in all phases of software
design and implementation, so the concept must become deeply ingrained.

\item
Programmers should know enough about common practice to avoid
reinventing the wheel.
Thus, programmers need to learn the commonly used data
structures, their related algorithms, and the most frequently
encountered design patterns\index{design pattern} found in programming.

\item
Data structures follow needs.
Programmers must learn to assess application needs first,
then find a data structure with matching capabilities.
To~do this requires competence in Principles~1, 2,
and~3.

\end{enumerate}

As I have taught data structures through the years, I have found that
design issues have played an ever greater role in my courses.
This can be traced through the various editions of this textbook by
the increasing coverage for design patterns and generic interfaces.
The first edition had no mention of design patterns.\index{design pattern}
The second edition had limited coverage of a few example patterns, and
introduced the
dictionary\index{dictionary}\index{abstract data type (ADT)}
\ifthenelse{\boolean{cpp}}{ADT and comparator classes.\index{comparator}}{}
\ifthenelse{\boolean{java}}{ADT.}{}
With the third edition, there is explicit coverage of some design
patterns that are encountered when programming the basic data
structures and algorithms covered in the book.

\paragraph{Using the Book in Class:}
Data structures and algorithms textbooks tend to fall into one of two
categories: teaching texts or encyclopedias.
Books that attempt to do both usually fail at both.
This book is intended as a teaching text.
I~believe it is more important for a practitioner to understand
the principles required to select or design the data structure that
will best solve some problem than it is to memorize a lot of textbook
implementations.
Hence, I~have designed this as a teaching text that covers most
standard data structures, but not all.
A few data structures that are not widely adopted are included
to illustrate important principles.
Some relatively new data structures that should
become widely used in the future are included.

Within an undergraduate program, this textbook is designed for use in
either an advanced lower division (sophomore or junior level) data
structures course, or for a senior level algorithms course.
New material has been added in the third edition to support its use in
an algorithms course.
Normally, this text would be used in a course beyond the standard
freshman level ``CS2'' course that often serves as the initial
introduction to data structures.
Readers of this book should typically have two semesters of the
equivalent of programming experience,
including at least some exposure to \Lang. %\index{\lang@\Lang}
Readers who are already familiar with recursion\index{recursion} will
have an advantage.
Students of data structures will also benefit from having first
completed a good course in Discrete
Mathematics.\index{discrete mathematics}
Nonetheless, Chapter~\ref{MathPre} attempts to give a reasonably
complete survey of the prerequisite mathematical topics at the level
necessary to understand their use in this book.
Readers may wish to refer back to the appropriate sections as
needed when encountering unfamiliar mathematical material.

A sophomore-level class where students have only a little background
in basic data structures or analysis (that is, background equivalent
to what would be had from a traditional CS2 course)
might cover Chapters 1-11 in detail,
as well as selected topics from Chapter~\ref{AdvTree}.
That is how I~use the book for my own sophomore-level class.
Students with greater background might cover Chapter~\ref{Intro},
skip most of Chapter~\ref{MathPre} except for reference, briefly cover
Chapters~\ref{AlgAnal} and~\ref{LSQ},
and then cover chapters 5-12 in detail.
Again, only certain topics from Chapter~\ref{AdvTree} might be covered,
depending on the programming assignments selected by the instructor.
A senior-level algorithms course would focus on Chapters 11 and 14-17.

Chapter~\ref{AdvTree} is intended in part as a source for
larger programming exercises.
I~recommend that all students taking a data structures course be
required to implement some advanced tree structure, or another dynamic
structure of comparable difficulty such as the
skip list\index{skip list@Skip List} or
sparse matrix\index{matrix!sparse} representations of
Chapter~\ref{ReList}.
None of these data structures are significantly more difficult to
implement than the binary search tree,\index{bst@BST} and any of them
should be within a student's ability after completing
Chapter~\ref{BinaryTree}.

While I have attempted to arrange the presentation in an order that
makes sense, instructors should feel free to rearrange the
topics as they see fit.
The book has been written so that once the reader has mastered
Chapters~1-6, the remaining material has relatively few dependencies.
Clearly, external sorting\index{sorting!external} depends on
understanding internal sorting\index{sorting!internal} and disk files.
Section~\ref{ParentPointer} on the UNION/FIND
algorithm\index{union/find@UNION/FIND}
is used in Kruskal's Minimum-Cost Spanning Tree
algorithm.\index{kruskal's algorithm@Kruskal's algorithm}
Section~\ref{SelfOrg} on
self-organizing lists\index{list!self-organizing} mentions the
buffer replacement schemes covered in
Section~\ref{BuffPool}.\index{buffer pool}
Chapter~\ref{AnalTech} draws on examples from throughout the book.
Section~\ref{SecHardProb} relies on knowledge of graphs.\index{graph}
Otherwise, most topics depend only on material presented earlier
within the same chapter.

Most chapters end with a section entitled ``Further Reading.''
These sections are not comprehensive lists of
references on the topics presented.
Rather, I~include books and articles that, in my opinion,
may prove exceptionally informative or entertaining to the reader.
In~some cases I~include references to works that should become
familiar to any well-rounded computer scientist.

\paragraph{Use of \Langsf:}
%\index{\lang@\Lang|(}
The programming examples are written in \Lang, but
I~do not wish to discourage those unfamiliar with \Lang\ from reading
this book.
I~have attempted to make the examples as clear as possible while
maintaining the advantages of \Lang.
\Lang\ is used here strictly as a tool to illustrate data structures
concepts.
In~particular, I~make use of \Lang's support for
hiding implementation details, including features such as classes,
private class members,
\ifthenelse{\boolean{java}}{and interfaces.}{}
\ifthenelse{\boolean{cpp}}{constructors, and destructors.}{}
These features of the language support the crucial concept of
separating logical design, as embodied in the abstract data type, from
physical implementation as embodied in the
data structure.\index{data structure!physical vs. logical form}

\ifthenelse{\boolean{cpp}}
{To keep the presentation as clear as possible, some
important features of \Lang\ are avoided here.
I~deliberately minimize use of certain features commonly
used by experienced \Lang\ programmers such as
class hierarchy,\index{object-oriented programming!class}
inheritance,\index{object-oriented programming!inheritance} and
virtual functions.\index{virtual function}
Operator\index{operator overloading} and function overloading
is used sparingly.
\C-like initialization syntax is preferred to some of the
alternatives offered by \Lang.}{}

\ifthenelse{\boolean{java}}
{As with any programming language, \Lang\ has both
advantages and disadvantages.
\Lang\ is a small language.
There usually is only one language feature to do something,
and this has the happy tendency of encouraging a programmer toward
clarity when used correctly.
In this respect, it is superior to \C\ or \LangCPP.
\LangJava\ serves nicely for defining and using most traditional data
structures such as lists and trees.
On the other hand, \Lang\ is quite poor when used to do file
processing, being both cumbersome and inefficient.
It is also a poor language when fine control of memory is required.
As an example, applications requiring memory
management, such as those discussed in Section~\ref{MemMan}, are
difficult to write in \Lang.
Since I wish to stick to a single language throughout the text, like
any programmer I must take the bad along with the good.
The most important issue is to get the ideas across, whether or not
those ideas are natural to a particular language of discourse.
Most programmers will use a variety of programming languages
throughout their career, and the concepts described in this book
should prove useful in a variety of circumstances.}{}

\ifthenelse{\boolean{java}}
{Inheritance, a key feature of object-oriented programming, is used
sparingly in the code examples.
Inheritance}{}
\ifthenelse{\boolean{cpp}}
{While the \Lang\ features mentioned above have valid design rationale
in real programs, they tend to obscure rather than enlighten the
principles espoused in this book.
For example, inheritance}{}
is an important tool that helps programmers avoid
duplication, and thus minimize bugs.\index{inheritance}
From a pedagogical standpoint, however, inheritance often makes
code examples harder to understand since it tends to spread the
description for one logical unit among several classes.
Thus, my class definitions only use inheritance where inheritance is
explicitly relevant to the point illustrated
(e.g., Section~\ref{PointerBin}).
This does not mean that a programmer should do likewise.
Avoiding code duplication and minimizing errors are important goals.
Treat the programming examples as illustrations of data structure
principles, but do not copy them directly into your own programs.

One painful decision I~had to make was whether to use \Gens\
in the code examples.
\ifthenelse{\boolean{java}}
{\GGens\ were not used in the first edition of this book.\index{generics}
But in the years since then, \Lang\ has matured and its use}{}
\ifthenelse{\boolean{cpp}}
{In the first edition of this book, the decision was to leave \Gens\
out as it was felt that their syntax
obscures the meaning of the code for those not familiar with
\Lang.\index{templates}
In the years following, the use of \Lang}{}
in computer science curricula has greatly expanded.
I now assume that readers of the text will be
familiar with \Gen\ syntax.
Thus, \Gens\ are now used extensively in the code examples.

My implementations are meant to provide concrete illustrations of data
structure principles, as an aid to the textual exposition.
Code examples should not be read or used in isolation from the
associated text because the bulk of each example's documentation is
contained in the text, not the code.
The code complements the text, not the other way around.
They are not meant to be a series of commercial-quality class
implementations.
If you are looking for a complete implementation of a standard data
structure for use in your own code, you would do well to do an
Internet search.

For instance, the code examples provide less parameter checking than is
sound programming practice, since including such checking would
obscure rather than illuminate the text.
Some parameter checking and testing for other constraints
(e.g., whether a value is being removed from an empty container)
is included in the form
of\ifthenelse{\boolean{java}}
{calls to methods in class \Cref{Assert}.}{}
\ifthenelse{\boolean{cpp}}
{a call to \Cref{Assert}.}{}
\index{assert@\Cref{assert}}
\ifthenelse{\boolean{cpp}}
{The inputs to \Cref{Assert} are a Boolean expression and a character
string.}{}
\ifthenelse{\boolean{java}}
{Method \Cref{Assert.notFalse} takes a Boolean expression.}{}
If~this expression evaluates to \FALSE, then a message is printed and
the program terminates immediately.
\ifthenelse{\boolean{java}}
{Method \Cref{Assert.notNull} takes a reference to class \Cref{Object},
and terminates the program if the value of the reference is \NULL.
(To be precise, they throw an
\Cref{IllegalArgument\-Exception}, which will
terminate the program unless the programmer takes action to handle
the exception.)}{}
Terminating a program when a function receives a bad parameter is
generally considered undesirable in real programs,
but is quite adequate for understanding how a data structure is
meant to operate.
In real programming applications, \Lang's exception handling features
should be used to deal with input data errors.
However, assertions provide a simpler mechanism for indicating
required conditions in a way that is both adequate for clarifying how
a data structure is meant to operate, and is easily modified into true
exception handling.
\ifthenelse{\boolean{cpp}}
{See the Appendix for the implementation of \Cref{Assert}.}{}

I~make a distinction in the text between ``\Lang\
implementations'' and ``pseudocode.''\index{pseudocode}
Code labeled as a \Lang\ implementation has actually been
compiled and tested on one or more \Lang\ compilers.
Pseudocode examples often conform closely to \Lang\ syntax, but
typically contain one or more lines of higher-level description.
Pseudocode is used where I~perceived a greater pedagogical
advantage to a simpler, but less precise, description.
%\index{\lang@\Lang|)}

\paragraph{Exercises and Projects:}
Proper implementation and analysis of data structures cannot be learned
simply by reading a book.
You must practice by implementing real programs,
constantly comparing different techniques to see what really works
best in a given situation.

One of the most important aspects of a course in data structures is
that it is where students really learn to program using pointers and
dynamic memory allocation, by implementing data structures such as
linked lists and trees.
It is often where students truly learn recursion.
In our curriculum, this is the first course where students do
significant design, because it often requires real data structures to
motivate significant design exercises.
Finally, the fundamental differences between memory-based and
disk-based data access cannot be appreciated without practical
programming experience.
For all of these reasons, a data structures course cannot succeed
without a significant programming component.
In our department, the data structures course is one of the most
difficult programming course in the curriculum.

Students should also work problems to develop their
analytical abilities.
I~provide over 450 exercises and suggestions for programming
projects.
I~urge readers to take advantage of them.

\paragraph{Contacting the Author and Supplementary Materials:}
A book such as this is sure to contain errors and have room for
improvement.
I~welcome bug reports and constructive criticism.
I~can be reached by electronic mail via the Internet at
{\tt shaffer@vt.edu}.
Alternatively, comments can be mailed to

\medskip

{\obeylines\parskip=-1pt
\qquad Cliff Shaffer
\qquad Department of Computer Science
\qquad Virginia Tech
\qquad Blacksburg, VA 24061
}

\medskip

The electronic posting of this book, along with a
set of lecture notes for use in class can be obtained
at
\begin{center}
\bfseries\url{http://www.cs.vt.edu/~shaffer/book.html}.
\end{center}
The code examples used in the book are available at the same site.
Online Web pages for Virginia Tech's sophomore-level data
structures class can be found at
\begin{center}
\bfseries\url{http://courses.cs.vt.edu/~cs3114}.
\end{center}

Readers of this textbook will be interested in our open-source, online
eTextbook project, OpenDSA ({\bfseries\url{http://algoviz.org/OpenDSA}}).
The OpenDSA project's goal is to ceate a complete collection of
tutorials that combine textbook-quality content with algorithm
visualizations for every algorithm and data structure, and a rich
collection of interactive exercises.
When complete, OpenDSA will replace this book.

This book was typeset by the author using \LaTeX.
The bibliography was prepared using \BibTeX.
The index was prepared using \MakeIndex.
The figures were mostly drawn with \Xfig.
Figures~\ref{RunTimeGraph} and \ref{HashLoad}
were partially created using \Mathematica.

% acknow.tex
% A Practical Introduction to Data Structures and Algorithm Analysis
% 3rd Edition: Shared between C++ and Java versions

\paragraph{Acknowledgments:}
It takes a lot of help from a lot of people to make a book.
I wish to acknowledge a few of those who helped to make this
book possible.
I apologize for the inevitable omissions.

Virginia Tech helped make this whole thing possible through sabbatical
research leave during Fall 1994, enabling me to get the project
off the ground.
My department heads during the time I have written the various
editions of this book, Dennis Kafura and Jack Carroll, provided
unwavering moral support for this project.
Mike Keenan, Lenny Heath, and Jeff Shaffer provided valuable
input on early versions of the chapters.
I also wish to thank Lenny Heath for many years of stimulating
discussions about algorithms and analysis (and how to teach both to
students).
Steve Edwards deserves special thanks for spending so much time
helping me on various redesigns of the \LangCPP\ and \LangJava\ code versions
for the second and third editions,
and many hours of discussion on the principles of program design.
Thanks to Layne Watson for his help with \Mathematica,
and to Bo Begole, Philip Isenhour, Jeff Nielsen, and Craig Struble for
much technical assistance.
Thanks to Bill McQuain, Mark Abrams and Dennis Kafura for answering
lots of silly questions about \LangCPP\ and \LangJava.

I am truly indebted to the many reviewers of the various editions of this
man\-u\-script.
For the first edition these reviewers included
J. David Bezek (University of Ev\-ans\-ville),
Douglas Campbell (Brig\-ham Young University), 
Karen Davis (University of Cincinnati),
Vijay Kumar Garg (University of Texas -- Austin),
Jim Miller (University of Kansas),
Bruce Maxim (University of Michigan -- Dearborn),
Jeff Parker (Agile Net\-works/\-Har\-vard),
Dana Richards (George Mason University),
Jack Tan (University of Hou\-s\-ton),
and Lixin Tao (Concordia University).
Without their help, this book would contain many more technical errors
and many fewer insights.

For the second edition, I wish to thank these reviewers:
Gurdip Singh (Kansas State University),
Peter Allen (Columbia University),
Robin Hill (University of Wyoming),
Norman Jacobson (University of California -- Irvine),
Ben Keller (Eastern Michigan University), and
Ken Bosworth (Idaho State University).
In addition, I wish to thank Neil Stewart and Frank J. Thesen for
their comments and ideas for improvement.

Third edition reviewers included Randall Lechlitner (University of
Houstin, Clear Lake) and Brian C. Hipp (York Technical College).
I thank them for their comments.

Prentice Hall was the original print publisher for the first and
second editions.
Without the hard work of many people there, none of this
would be possible.
Authors simply do not create printer-ready books on their own.
Foremost thanks go to Kate Hargett, Petra Rector, Laura Steele, and
Alan Apt, my editors over the years.
My production editors, Irwin Zucker for the second edition,
Kathleen Caren for the original \LangCPP\ version,
and Ed DeFelippis for the \LangJava\ version,
kept everything moving smoothly during that horrible rush at the end.
Thanks to Bill Zobrist and Bruce Gregory (I think) for getting me
into this in the first place.
Others at Prentice Hall who helped me along the way include Truly
Donovan, Linda Behrens, and Phyllis Bregman.
Thanks to Tracy Dunkelberger for her help in returning the copyright
to me, thus enabling the electronic future of this work.
I am sure I owe thanks to many others at Prentice Hall for their
help in ways that I am not even aware of.

I am thankful to Shelley Kronzek at Dover publications for her faith
in taking on the print publication of this third edition.
Much expanded, with both \LangJava\ and \LangCPP\ versions, and many
inconsistencies corrected, I am confident that this is the best
edition yet.
But none of us really knows whether students will prefer a free online
textbook or a low-cost, printed bound version.
In the end, we believe that the two formats will be mutually
supporting by offering more choices.
Production editor James Miller and design manager Marie Zaczkiewicz
have worked hard to ensure that the production is of the highest
quality.

I wish to express my appreciation to Hanan Samet for teaching me about
data structures.
I learned much of the philosophy presented here from
him as well, though he is not 
responsible for any problems with the result.
Thanks to my wife Terry, for her love and support, and to my daughters 
Irena and Kate for pleasant diversions from working too hard.
Finally, and most importantly, to all of the data structures students
over the years who have taught me what is important and what should
be skipped in a data structures course, and the many new insights
they have provided.
This book is dedicated to them.
\bigskip
{\obeylines\parskip=0pt\parindent=0pt
\hfill Cliff Shaffer
\hfill Blacksburg, Virginia
}

